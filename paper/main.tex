\documentclass[10pt]{article}
\usepackage{stat}
\usepackage{graphicx}
\usepackage[margin=1in]{geometry}
\usepackage{natbib}
\bibliographystyle{plainnat}

\title{On the \textit{Mechanistic Stability} of Language Models}
\date{\today}

\setlength\parindent{0pt}%

\begin{document}

\ifthenelse{\isundefined{\showlinenum}}{
}{
\linenumbers
}
\maketitle


Herein, we introduce a notion which we call \textit{mechanistic stability} which
characterizes the sensitivity of a models' prediction criteria with respect to
perturbations of the input data. 

\section{What are we doing?}
For any task $T$, we care about whether a model is mechanistically stable on $T$.
That is, is it using the same causal graph to perform inference? The main result
that we want to prove in our paper is that mechanistic stability is good for
the model in terms of both generalization and robustness. Throughout our paper,
we assume the supervised learning setting. 

We denote $\cX, \cY$ to the universe of all possible input and outputs. Then, 
a task is simply a distribution over the product of $\cX,\cY$. Concretely,

\begin{defn}[Task]
A task $T$ is a distribution over $\cX \times \cY$. We denote this
distribution as $\cD_{\cX\times\cY}$. In the literature, this is also
called the data distribution. 
\end{defn}
Stability is a measure of change in ``output'' versus changes in ``input.'' Here,
the terms input/output are used loosely. Mechanistic stability is when the output
is the mechanism of our model and where the input is across partitions of 
$\cX \times \cY$ according to 
\begin{defn}[Permissible Partition]
For a given task $T$ with distribution $\cD_{\cX\times\cY}$, a $T$-partition,
denoted by $\cS$, is a countable collection of measurable subsets of 
$\cX\times \cY$ such that $\coprod_{s\in \cS} s = \cX \times \cY$ 
and $\P_{\cD_{\cX\times\cY}}[s] > 0$ for all $s \in \cS$. 
\end{defn}
Now that we have a very loose notion of what the ``input'' looks like, we
need to measure what big changes in our output corresponds to. I'm not too
sure exactly what this looks like yet and we probability need theory from
the causality and category theory papers.

After we perform mechanism extraction, we need to assume that by virtue
of this being the mechanism, this is essentially the function that we
care about. So, if two mechanism are exactly the same, then their 
functions must be the same. Is this the case in the category-theoretic
sense? In other words, if two causal models are equivalence in the category-
theoretic sense, how different can they be in terms of their output?

Trivially, mechanistic equivalence should give us robustness. 
The main issue that we care about here is generalization. In the sense
that

\begin{defn}
Let $\rho(\cdot,\cdot)$ denote a metric on the set of all causal 
graphs with respect to
a distribution $\cD_{\cX\times\cY}$.
\end{defn}
At the core of our paper, we are making the following assumption. This is 
common for many papers in mechanistic interpretability:
\begin{asmp}
The extracted mechanisms of a neural network is its causal graph. 
\end{asmp}
Now, we are ready to define mechanistic stability. Let $M$ be a model that
represents a stochastic mapping from $\cX\to\cY$. From now on, we shall slightly
abuse notation and refer to both the model and its mechanism as $M$. It should
be clear from context what exactly we mean. 
\begin{defn}
We say that $M$ is $\eps$-mechanistically stable on a task $T$ if 
\[
\sup_{\cS: T\text{-partition}}\E_{p_1,p_2 \in \cS}
\rho(M_{p_1}, M_{p_2}) < \eps,
\]
where $M_{p_i}$ is the mechanism of $M$ when restricted to
the truncated distribution over $p_i$. 
\end{defn}

At this point, there are still some questions that need to be resolved.
Firstly, how do we define $\rho(\cdot,\cdot)$ and does it need to be an
actual metric, for example maybe we can only determine equivalence
up to some isomorphism? Secondly, does this definition of all possible
permissible partitions make sense? Is it too weak? That is, are there
pathological partitions across whose stability we do not care about?

\section{Main results}
Herein, is a wish list of the main results that we wish to achieve.
\begin{thm}
Mechanistic stability $\Rightarrow$ in-distribution 
robustness, assuming that both $\cX, \cY$ are metric spaces. 
We also need that the task distribution $\cD_{\cX\times \cY}$ is smooth.
\end{thm}
\begin{thm}
Mechanistic stability $\Rightarrow$ in-distribution generalization.
\end{thm}
Maybe for proving this latter theorem, we can look at some proofs relating regularization.
What is a notion of in-distribution generalization that makes sense to look
at here. Can we say something stronger maybe? What about out-of-distribution?
like distribution shift?

\section{Casual Equivalence: A Category-Theoretic Perspective}
Herein, we describe briefly how to understand the equivalence between two mechanisms.
Our fundamental assumption is that the extracted mechanism of a model is 
a causal graph.
A causal graph is a functor from $F: \textsf{Syn} \to \textsf{Stoch}$


\section{Related Literature}
What is the relationship between our notion of mechanistic stability and 
the more general notion of algorithmic stability? Is it potentially the same
as algorithmic stability applied to in-context learning?

\textbf{Algorithmic stability.} What is the relationship between algorithmic
stability and the notion of mechanistic stability that we are defining, both
in terms of the generalization bounds that are guaranteed with algorithmic
stability and in terms of the concept itself (how are we measuring
the distance between two learned hypotheses?)

\textbf{Mechanistic interpretability.} Recently, there has a been a push to interpret
neural networks by uncovering their
\textit{mechanisms}. A mechanism of a neural network with respect to some task
is a minimal subgraph of its computational graph that wholly characterizes the
network's behavior on this task~\citep{wang_interpretability_2022}. After exposing
the driving mechanisms and reverse-engineering its network into human-interpretable
components, this facilitates the understanding and improvement of models.  The general
convention has been to manually uncover the mechanisms of models. However, recent work~\citep{conmy_towards_2023}
aims to automate this process to promote work on interpretability across the field. Others 
have used concept bottlenecks~\citep{tan_interpreting_2023}, attention weight 
visualization~\citep{galassi_attention_2020}, probing feature representations~\citep{lundberg_a_2017}, and 
counterfactuals~\citep{wu_polyjuice_2021} to attempt to uncover LLMs' black-box nature. These techniques, 
though, are all domain-dependent and thus, do not allow for general interpretability of LLMs.

Our paper aims to create a generalizable metric that can be used to interpret and measure the performance 
of models across various domains and models.

\textbf{Graphical models.} Probably need to rethink the title here, we want
to explain the relationship between directed acyclic graphs, causal graphs,
and the mechanisms that we are extracting from the neural network. Our
underlying assumption is that all of these things are the same.

\section{Experimental Methods}
Can we show mechanistic stability and instability on a host of tasks?
\begin{enumerate}
\item IOI
\item Colored Objects
\item Arithmetic
\item General algorithmic tasks
\item General reasoning tasks (this and the previous task would benefit
from increased test time compute and enhanced supervision; so we can we
somehow increase stability by increasing one of these factors?)
\item General knowledge tasks (would not benefit from chain-of-thought or more
test time compute)
\end{enumerate}
What is the effect of scale on mechanistic stability? Or conversely, the benefits
that we see in terms of performance as a result of increased scale, does this
come from increased mechanistic stability?


\bibliography{references}

\newpage
\section{Category-Theoretic Causality}

\end{document}
